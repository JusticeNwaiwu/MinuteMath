\author{}
\title{SE 3354.003 - Software Engineering1}
\documentclass[12pt,letterpaper,oneside]{article}

% Loaded packages from ./requirements.sty
\usepackage{requirements}

% Metadata
\usepackage[pdftex,
	pdftitle={SE 3354.003 Project Requirements},
	pdfsubject={Software Engineering},
	pdfproducer={LaTeX with hyperref},
pdfcreator={pdflatex}]{hyperref}

% Headers and foots
\pagestyle{fancy}
\fancyhf{}
\lhead{SE 3354.003}
\chead{Project Requirements}
\rhead{MinuteMath}

\begin{document}
%===============================================================================
\begin{enumerate}
	\item
		The application shall contain tutorials to teach various mathematical concepts
		\begin{enumerate}
			\item
				Each tutorial shall be comprised of five attributes: name, category, difficulty, content, completion
				\begin{enumerate}
					\item
						A tutorial's name shall describe the material to be learned in the content
					\item
						A tutorial's category shall describe  which mathematical subject the content falls under
					\item
						A tutorial's difficulty shall display the level of difficulty of the tutorial
						\begin{enumerate}
							\item
								A tutorial's difficulty is relative to that of another
						\end{enumerate}
					\item
						A tutorial's content shall be comprised of learning material and example problems
						\begin{enumerate}
							\item
								Learning material shall be described in steps
							\item
								Example problems shall be worked out and solved within the content
						\end{enumerate}
					\item
						A tutorial's completion shall display how much of a tutorial has been completed
				\end{enumerate}
			\item
				All tutorials shall be sorted by name, category, and difficulty
				\begin{enumerate}
					\item
						Tutorials shall be sorted by category
					\item
						Tutorials of the same category shall then be sorted by difficulty
					\item
						Tutorials of the same difficulty shall then be sorted by name
				\end{enumerate}
		\end{enumerate}

	\item	% Games
		The application shall contain games to reinforce learning of mathematical concepts
		\begin{enumerate}
			\item	% What is a game?
				A game shall be composed of a series of randomly generated mathematical problems for the user to solve
				\begin{enumerate}
					\item % What is a problem?
						Each problem shall be composed of two operands and an operator
				\end{enumerate}
			\item	% Game types
				The operator used in generating a game's problems shall be selected by the user
				\begin{enumerate}
					\item % Types of operands
						The possible operators shall be addition, subtraction, multiplication and division
				\end{enumerate}
			\item	% Number generation
				The games' operands shall be randomly generated
			\item	% Options
				Each game shall have user-configurable parameters
				\begin{enumerate}
					\item	% Duration
						The game duration shall be configurable by the user
					\item	% Number of questions
						The number of problems shall be configurable by the user
					\item	% Number range
						The range of numbers shall be configurable by the user
						\begin{enumerate}
							\item % Addition and subtraction
								The specified range shall limit the operands when the operator is addition, subtraction or multiplication
							\item % Multiplication and division
								The specified range shall limit the dividend when the game type is division
						\end{enumerate}
				\end{enumerate}
			\item	% End game
				A game shall end when the specified duration or number of questions has been met or if the game is manually ended by the user
		\end{enumerate}

	\item	% Feedback
		The application shall generate feedback for the user
		\begin{enumerate}
			\item	% What data is used?
				Data regarding the user's performance shall be collected during each game
				\begin{enumerate}
					\item	% Number of missed
						The number of problems answered correctly during a game shall be recorded
					\item	% Number of correct
						The number of problems answered incorrectly during a game shall be recorded
					\item	% Time taken
						The amount of a user spent in a game shall be recorded
				\end{enumerate}
			\item
				Each game's collected data shall be analyzed
				\begin{enumerate}
					\item	% Accuracy
						The user's accuracy shall be calculated
					\item	% Speed
						The user's speed shall be calculated
				\end{enumerate}
			\item	% Performance over time
				The user's performance shall be tracked over time
		\end{enumerate}

	\item	% User profile
		The application shall store feedback in a user profile

	\item	% GUI
		The application shall implement a graphical user interface (GUI)
		\begin{enumerate}
			\item	% Main menu
				The appplication shall show a main menu on initial launch
				\begin{enumerate}
					\item
						The main menu shall contain buttons to user profile, games, and tutorials
				\end{enumerate}
			\item	% Tutorials
				The GUI shall contain a tutorial section
				\begin{enumerate}
					\item	% Categorized tutorials
						The tutorial screen shall display high-level tutorial categories for the user to choose from
					\item	% Tutorial drop-downs
						The tutorial categories shall display drop-downs of their contained tutorials upon being tapped
					\item	% Tap to enter
						The user shall be taken to the appropriate tutorial upon tapping a contained tutorial
						\begin{enumerate}
							\item
								The tutorial shall be a text-based activity with scrolling
							\item
								The user shall be able to navigate back to other menus from the tutorial
						\end{enumerate}
				\end{enumerate}
			\item
				The GUI shall contain a games section
				\begin{enumerate}
					\item
						The GUI shall provide a game select screen for the user to select from the application's games
						\begin{enumerate}
							\item
								The game select screen shall display categories for the user to chose from
							\item
								The user shall be taken to the game options screen upon tapping a catagory
						\end{enumerate}
					\item
						The GUI shall provide a game options screen for user to configure the game parameters
						\begin{enumerate}
							\item	% Duration
								The game options screen shall display options for the user to manipulate game time duration
							\item	% Number of questions
								The game options screen shall display options for the user to specify the number of questions
							\item	% Scores
								The game options screen shall display options for the user to specify whether or not scores will be shown during the game
							\item
								The user shall be taken to the appropriate game upon tapping a game start button
						\end{enumerate}
					\item
						The GUI shall provide a way for the user to play the application's games
						\begin{enumerate}
							\item
								The in-game screens shall contain a button to end the current game
						\end{enumerate}
				\end{enumerate}
			\item
				% User profile
				The GUI shall contain a section for viewing the user's statistics
				\begin{enumerate}
					\item
						The statistics screen shall display high-level statistics categories for the user to view
						\begin{enumerate}
							\item
								The high-level categories shall be displayed in a scrollable, horizontal bar
						\end{enumerate}
					\item
						The information within each category shall be presented graphically
				\end{enumerate}
		\end{enumerate}

\end{enumerate}
%===============================================================================
\end{document}

